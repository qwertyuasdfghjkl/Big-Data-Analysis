\documentclass[11pt]{article}
%Created on 4/10/07.

%%%extra package
%\usepackage{imsart}


%%%%%%%%%%%%%%%%%%%%%%%%%%
%%%%%%%%%%%%%%%%%%%%%%%%%%
%%%%%%%%%%%%%%%%%%%%%%%%%%
\usepackage{amsmath,amssymb}

%% Please use the following statements for
%% managing the text and math fonts for your papers:
\usepackage{times}
%\usepackage[cmbold]{mathtime}
\usepackage{bm}
%\usepackage{natbib}
\usepackage{algorithm}
\usepackage{xargs}[2008/03/08]
\usepackage{amssymb}
\usepackage{mathrsfs}
\usepackage{graphicx}
\usepackage{rotating,subfigure}
\usepackage[flushleft]{threeparttable}
\usepackage{multirow}
\usepackage{enumitem} %%%enumitem

%%%STAR micro
\usepackage{star}


%\usepackage[usenames,dvipsnames,svgnames,table]{xcolor}
\usepackage[colorlinks,
linkcolor=blue,
anchorcolor=blue,
citecolor=blue
]{hyperref}

%%%%operators
\def\skeptic{{\sc skeptic}}
\newcommand{\sgn}{\mathop{\mathrm{sign}}}
\providecommand{\bnorm}[1]{\big\|#1\big\|}
\providecommand{\enorm}[1]{| \! | \! |#1| \! | \! |}
\providecommand{\bemnorm}[1]{\big| \! \big| \! \big|#1\big| \! \big| \! \big|}


%%%%My macros 
\newcommand*{\Sc}{\cS^{\perp}}
\newcommand*{\Ac}{\cA^{\perp}}
\newcommand*{\supp}{\mathrm{supp}}
%\usepackage{cite}

\newcommand \rw{\mathrm{w}}
%%%%Definition of Equation environment
\def\T{{ \mathrm{\scriptscriptstyle T} }}
\def\v{{\varepsilon}}
  
%%%%Definition of Equation environment
\def\##1\#{\begin{align}#1\end{align}}
\def\$#1\${\begin{align*}#1\end{align*}}

%%%%notation macros
\newcommand{\rF}{\textnormal{F}}
\renewcommand{\tr}{\textrm{trace}}

%%%%Definition of Operators
\newcommand {\vecc}{\textnormal {vec}}
\def\T{{ \mathrm{\scriptscriptstyle T} }} %%%transpose operator

%%%%Definition of Roman Numbers
\newcommand{\Rom}[1]{\text{\uppercase\expandafter{\romannumeral #1\relax}}}


%%%comment
\newcommand{\scolor}[1]{{\color{magenta}#1}}
\newcommand{\sscomment}[2]{\scolor{ #1}\marginpar{\tiny\scolor{SQ:\ #2}}}
\newcommand{\scomment}[1]{\scolor{$\dagger$}\marginpar{\tiny\scolor{SQ:\ #1}}\hspace{-3pt}}


%%%for adobe time roman font
\usepackage{txfonts}


%%%margin and textwidth
\usepackage{geometry}
 \geometry{
 a4paper,
 %total={170mm,257mm},
 left=31mm,
 top=30mm,
 }
\textwidth=6in


%%%baseline stretch
\renewcommand{\baselinestretch}{1.2}

%%%%%%%%%%%%%%%%%%%%%%%%%%
%%%%%%%%%%%%%%%%%%%%%%%%%%
%%%%%%%%%%%%%%%%%%%%%%%%%%

%%%%Macros for homeworks
\renewcommand*{\proofname}{Solution}
\theoremstyle{mytheoremstyle}
\ifx\question\undefined
\newtheorem{question}{Question}
\fi


\begin{document}

\thispagestyle{empty}

%\draftnotice

\begin{center}
\bf\large Tutorial 1: Introduction to LaTex
\end{center}

\noindent
Yicheng  %%% FILL IN LECTURER (if not RS)
\hfill
Jan 21, 2021           %%% FILL IN LECTURE DATE HERE

\noindent
\rule{\textwidth}{1pt}

\medskip

%%%%%%%%%%%%%%%%%%%%%%%%%%%%%%%%%%%%%%%%%%%%%%%%%%%%%%%%%%%%%%%%
%% BODY OF SCRIBE NOTES GOES HERE
%%%%%%%%%%%%%%%%%%%%%%%%%%%%%%%%%%%%%%%%%%%%%%%%%%%%%%%%%%%%%%%%

\section{Install Your LaTex}

\subsection{Install a LaTex Compiler}
Find it via \url{https://www.latex-project.org/get/}

\subsection{Install a LaTex Editor}
For Windows: WinEdt, TeXMaker, TeXstudio, etc;\\
For Mac: TeXShop, MacTex, etc.

\section{Fonts Selection}
\subsection{Text-Mode Fonts}
Text-mode: Given random samples .... \\
\textbf{Given random samples ....} {\bf Given random samples ....}\\
\textit{Given random samples ....} {\it Given random samples ....}\\

{\tiny size} {\scriptsize size} {\footnotesize size} {\small size} size {\large size} {\Large size} {\LARGE size} {\huge size} {\Huge size}

\subsection{Math-Mode Fonts}
Math mode: $X_1,\cdots,X_n$ ... \\
$ABC$, $\mathbf{ABC}$,  $\mathcal{ABC}$,  $\mathbb{ABC}$

\section{Format}
\subsection{Structure}
Sections $\longrightarrow$ Subsections $\longrightarrow$ Subsubsections

\subsubsection{subsubsection}

NO ``subsubsubsections''

\subsection{New lines and paragraphs}
Line1\\
Line2\\
Line3

Par1

Par2

Par3

\noindent Par4

\vspace{0.2in}
Par5


%\iffalse
\section{Tables}
A table through {\color{blue}tabular} environment.
\begin{table}[h]
\centering
\begin{tabular}{c|c|c|c}
\hline A & B & C & D\\
\hline a &b &c &d\\
\hline 1 &2 & 3 &4\\
\hline
\end{tabular}
\end{table}
%\fi

\section{Figures}

\begin{figure}[h]
  \centerline{\includegraphics[scale=0.2]{circle.pdf}}
  \caption{Big Data Science Loop}
  \label{circle}
\end{figure}

By Figure~\ref{circle}, we can ... 

\newpage

\section{Some Frequently-Used Mathematical Notations}

\begin{align*}
&\alpha,\ \beta,\theta, \lambda,\  a_1,\cdots,a_n,\ b_1\ldots b_n,\ \{x_i\}_{1\le i\le n},\  X\sim \mathcal{N}(0,1)\\
&X_n\rightarrow X,\ X_n\overset{D}\longrightarrow X,\ a\geq b,\ a\leq b, a\neq b\\
&\sum_{i=1}^n a_i,\ \prod_{i=1}^n a_i,\ \lim_{n\rightarrow \infty}a_n,\ \int_a^b f(x)dx 
\end{align*}


\begin{align}
&\alpha,\ \beta,\theta, \lambda,\ a_1,\cdots,a_n,\ b_1\ldots b_n,\ \{x_i\}_{1\le i\le n},\  X\sim \mathcal{N}(0,1)\label{eq1}\\
&X_n\rightarrow X,\ X_n\overset{D}\longrightarrow X,\ a\geq b,\ a\leq b, a\neq b\\
&\sum_{i=1}^n a_i,\ \prod_{i=1}^n a_i,\ \lim_{n\rightarrow \infty}a_n,\ \int_a^b f(x)dx 
\end{align}


\begin{gather}
\alpha,\ \beta,\theta, \lambda,\ a_1,\cdots,a_n,\ b_1\ldots b_n,\ \{x_i\}_{1\le i\le n},\  X\sim \mathcal{N}(0,1)\\
X_n\rightarrow X,\ X_n\overset{D}\longrightarrow X,\ a\geq b,\ a\leq b, a\neq b\\
\sum_{i=1}^n a_i,\ \prod_{i=1}^n a_i,\ \lim_{n\rightarrow \infty}a_n,\ \int_a^b f(x)dx 
\end{gather}

By the equation~\eqref{eq1}, we can get that ... 



\section{An Example}

\begin{question}[Maximum Likelihood Estiamtor (MLE) and Asymptotic Normality (20 points)]
Maximum likelihood is one of the most fundamental principals in parameter estimation. Suppose we have $n$ i.i.d. random samples $\left\{X_i\right\}_{i=1}^n$ that have probability density function $p_\theta(x)$. We are interested in estimating the parameter $\theta$. Denote the correspondent MLE by $\hat \theta_n$. In the lecture, we have known that under some regularity conditions, the MLE enjoys the asymptotic normality
\begin{align}
\sqrt n (\hat \theta_n-\theta)\overset{D}\longrightarrow N(0,\frac{1}{I(\theta)}),
\end{align}
where
\begin{align*}
I(\theta):=\mathbb{E}\left(-\frac{\partial^2}{\partial \theta^2}\log p_\theta(X)\right)
=-\int_{\mathcal X}\left(\frac{\partial^2}{\partial \theta^2}\log p_\theta(x)\right)p_\theta(x)dx
\end{align*}
is the Fisher information and $\mathcal{X}$ is the range of $X_i$.
\end{question}



%%%%%%%%%%%%%%%%%%%%%%%%%%%%%%%%%%%%%%%%%%%%%%%%%%%%%%%%%%%%%%%%

\end{document}